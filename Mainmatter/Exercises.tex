

The purpose of these exercises is to make the reader familiar with Atlan. To get a feel for the language. They should not be seen as a test of the knowledge of the reader but rather as a guide to get your bearings in Atlan.  


\noindent \textbf{Exercise 10.1 -- Creating a basic possessive sentence \& writing your own name} 

In this first exercise we will be translating the sentence “My name is …" this exercise will demonstrate how to combine words in Atlan to create new words in a simple and familiar context. 

We will start by creating the Atlan word for “my”. To help you along I will give you the Atlan translation for “I”; EJ.AM \ej\am. “I” is composed of EJ \ej meaning person, and AM \am; 1st removed: speaker. The Atlan word for “my” can easily be made from this by adding the possessive prefix TA \ta. This leaves us with TA.EJ.AM \ta\ej\am.  

And what is the Atlan translation of “name”? And “is”? (Note: “is” is a predicate and comes with a marker). Remember that the subject of an Atlan word always comes first in a sentence, so the word order would be: Name my is [Name]. 

Now comes the easiest part of the sentence, your name. To write your name in Atlan all you must do is transliterate your name into Atlan’s set of 14 sounds and put a cartouche around your name. %Now, what would be the Atlan translations of “My name is …"? 
\begin{itemize}
	\item[(i)] Write your name in Atlan:
	\vspace{0.3cm}
	\dotfill
\end{itemize}

\noindent \textbf{Exercise 10.2 -- Creating basic active sentences in the present simple tense} 

Now that we know how to create a basic possessive sentence, we are going to look at how to form basic active sentences. Like, “I am walking” and “He is writing”.  
\begin{itemize}
\item[(i)] What is the Atlan translation for “I walk”, “You walk” and “He walks”? (Note: present tense has no need for a marker).  
\end{itemize}

\noindent And what is the English translation of the following sentences? 
\begin{itemize}
    \item[(ii)] EJ.AM TU.LIK EK.POK \ej\am \tu\lik \ek\pok  

    \item[(iii)] EJ.AJ TU.JIL EK.FIL\ej\aj \tu\jil \ek\fil  

    \item[(iv)] EJ.AM.ON TU.KOS.TOJ \ej\am\on \tu\kos\toj 
\end{itemize}

\noindent \textbf{Exercise 10.3 -- Creating basic sentences in the past simple tense} 

Now we move to the past simple tense. Translate the following English sentences to Atlan. 
\begin{itemize}
    \item[(i)]I worked. 

    \item[(ii)]We worked. 

    \item[(iii)]They worked. 

    \item[(iv)]You played yesterday. 
\end{itemize}
 

\noindent \textbf{Exercise 10.4 -- Numbers} 

Atlan’ number system can be used as a ten-base system and as a twelve-base system. To see the difference between these systems you can look at how to spell your name in both ten-base and twelve-base. We will begin with exercises that use the ten-base system as this will probably be more familiar. 

\noindent Translate the following sentences using a ten-base number system:


\begin{itemize}
    \item[(i)]I have three fish. 

    \item[(ii)]I have eleven fish. 

    \item[(iii)]I have a thousand fish. 
\end{itemize}
\noindent Translate the following sentences using a duodecimal (twelve-base) number system:
\begin{itemize}
    \item[(iv)]I have three fish. 

    \item[(v)]I have eleven fish. 

    \item[(vi)]I have a thousand fish. 

    \item[(vii)]I have a 1.728 fish.  
\end{itemize}
 

 
\pagebreak
 

 

\section{Solutions}
\small

\noindent{\bf 10.1} -  “My name is …" in Atlan would be; NA TA.EJ.AM SI [NAME] \na \ta\ej\am \si \cartouche{NAME}. 

 

\noindent{\bf 10.2}
\begin{itemize}
    \item[(i)]I walk = AM.TU.TOM \am\tu\tom 

    \item[(ii)]You walk = UN.TU.TOM \un\tu\tom 

    \item[(iii])He walks = AJ.TU.TOM \aj\tu\tom 
\end{itemize}
 

    %EJ.AM TU.LIK EK.POK \ej\am \tu\lik \ek\pok = I write a book. 
    %EK \ek indicates that POK \pok is the object of the sentence. 

    %EJ.AJ TU.JIL EK.FIL\ej\aj \tu\jil \ek\fil = He drives a vehicle. 

    %EJ.AM.ON TU.KOS.TOJ \ej\am\on \tu\kos\toj= We eat. 
    %\on indicates that \am is plural making it “we” instead of “I”. \kos (consume) combined with \toj (solid) means eat. 

\noindent {\bf 10.3}
\begin{itemize}
    \item[(i)]  EJ.AM PA.TU.KAN \ej\am \pa\tu\kan 

    \item[(ii)] EJ.AM.ON PA.TU.KAN \ej\am\on \pa\tu\kan 

    \item[(iii)]EN.AJ.ON PA.TU.KAN \ej\aj\on \pa\tu\kan 

    \item[(iv)]EJ.UN.ON PA.TU.LAJ ET.JAN.POP \ej\un\on \pa\tu\laj  \et\jan\pop 
\end{itemize}

\noindent \textbf{10.3.4 Numbers} 
\begin{itemize}
    \item[(i)]EJ.AM TU.TA MIS.UP - \ej\am \tu\ta \mis\up 

    \item[(ii)]EJ.AM TU.TA MIS.JI.IP - \ej\am \tu\ta \mis\ji\ip 

    \item[(iii)]EJ.AM TU.TA MIS.NU - \ej\am \tu\ta \mis\nu 

\item[(iv)] EJ.AM TU.TA MIS.UP  - \ej\am \tu\ta \mis\up 

    \item[(v)] EJ.AM TU.TA MIS.JO - \ej\am \tu\ta \mis\jo 

    \item[(vi)] EJ.AM TU.TA. MIS.UK.JO.IK - \ej\am \tu\ta \mis\uk\jo\ik 

    \item[(vii)] EJ.AM TU.TA MIS.NU \ej\am \tu\ta \mis\nu
\end{itemize}
