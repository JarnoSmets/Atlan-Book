\documentclass[12pt,a5paper]{article}
\usepackage{../Atlan}
\usepackage[bmargin=30pt]{geometry}
\usepackage{multicol}

\pagestyle{empty}
\newcommand{\restorecorps}{\renewcommand{\corpsgrootte}{20pt}}

\DefineLetter{\Atlann}{\DeclareStroke{\RightDiagonal}}

\begin{document}
\begin{center}
\bf \Huge Atlan
\end{center}

\noindent With the world being more globally connected than ever before, an International Auxiliary Language (IAL) can be a tool for individuals from diverse backgrounds to connect, collaborate, and understand each other. We, students from the University of Utrecht, embarked on an audacious mission: to construct such an International Auxiliary Language. In a whirlwind of only a couple of weeks, we birthed Atlan: a philosophical language based on the core principles of neutrality, unambiguity, and simplicity.  
\phantom{.}\\

 \noindent This book wants to take the reader on an odyssey with us in constructing this IAL. Analyze with us the fundamental principles of language and the complexities of phonological categories. Learn with us the original writing system, the cleverly crafted ontology, and the lexicon that is generated with data from languages worldwide. Follow us as we overcome barriers, transcend linguistic limitations, and in the process, unite the world in our words. 

\vspace{0.1cm}
\begin{center} 



	\parbox{0.8\textwidth}{%
\small
Jep Antonisse \hfill \ej \na \cartouche{\jep  \an\Atlanto\Atlanni\se}

Niek Elsinga \hfill \ej \na \cartouche{\nik  \el\Atlansin\ka}

\DefineLetter{\Atlans}{\DeclareStroke{\BigSW}}
Max Geraedts \hfill \ej \na \cartouche{\mak\Atlans  \ke\lat\Atlans}

%\columnbreak 

Stijn Janssens \hfill \ej \na \cartouche{\Atlans\tej\Atlann  \jan\sen\Atlans}

Jonathan Roose \hfill \ej \na \cartouche{\jo\na\Atlantan  \lo\se}

Jarno Smets \hfill \ej \na \cartouche{\jal\no\ \Atlans\met\Atlans}
	}%




\end{center}
\end{document}
